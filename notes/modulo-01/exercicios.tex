\section{Exercícios complementares}%

\begin{exercise}
Na análise de concessão de empréstimos, uma variável potencialmente
importante é a renda da pessoa. O gerente de um banco coleta uma base de
dados de seus correntistas e extrai a variável ``renda mensal (R\$)'' para 50
pessoas. Embora se trate de uma variável quantitativa, deseja realizar uma
análise por meio de tabela de frequências. Neste sentido, pede-se:
\begin{enumerate}[label=\alph*)]
    \item Classifique os correntistas em faixas de renda, sendo: 0-2.000;
    2.001-4.000; 4.001-6.000; 6.001-8.000; 8.001-10.000 e 10.001-12.000.
    \item Em seguida, elabore a tabela de frequências para as faixas de renda
    acima. O banco de dados está na planilha Lista de Exercício Complementares:
    aba Exercício 1.
\end{enumerate}
\end{exercise}

\begin{resolucao}
A \Cref{tab:ex_01a} traz, para cada observação, a renda mensal da
pessoa e a respectiva faixa de renda, de acordo com os critérios
solicitados. A tabela de frequências \ref{tab:ex_01b} refere-se a essas
faixas de renda.
\begin{table}[H]
    \centering
    \begin{tabular}{cccccc}
        \toprule
        Obs & Renda (R\$) & Faixa          & O  & R         & F \\
        \midrule
        1   & 2.894,00    & 2.001--4.000   & 26 & 7.665,00  & 6.001--8000    \\
        2   & 3.448,00    & 2.001--4.000   & 27 & 3.890,00  & 2.001--4.000   \\
        3   & 1.461,00    & 0--2.000       & 28 & 6.590,00  & 6.001--8000    \\
        4   & 2.224,00    & 2.001--4.000   & 29 & 1.241,00  & 0--2.000       \\
        5   & 2.501,00    & 2.001--4.000   & 30 & 1.720,00  & 0--2.000       \\
        6   & 1.100,00    & 0--2.000       & 31 & 2.556,00  & 2.001--4.000   \\
        7   & 3.560,00    & 2.001--4.000   & 32 & 4.730,00  & 4.001--6000    \\
        8   & 5.511,00    & 4.001--6000    & 33 & 4.745,00  & 4.001--6000    \\
        9   & 2.901,00    & 2.001--4.000   & 34 & 8.550,00  & 10.001--12.000 \\
        10  & 10.128,00   & 10.001--12.000 & 35 & 3.860,00  & 2.001--4.000   \\
        11  & 1.855,00    & 0--2.000       & 36 & 11.320,00 & 10.001--12.000 \\
        12  & 3.161,00    & 2.001--4.000   & 37 & 6.125,00  & 6.001--8000    \\
        13  & 8.630,00    & 10.001--12.000 & 38 & 5.606,00  & 4.001--6000    \\
        14  & 6.201,00    & 6.001--8000    & 39 & 3.250,00  & 2.001--4.000   \\
        15  & 4.130,00    & 4.001--6000    & 40 & 1.500,00  & 0--2.000       \\
        16  & 2.736,00    & 2.001--4.000   & 41 & 9.216,00  & 10.001--12.000 \\
        17  & 4.448,00    & 4.001--6000    & 42 & 4.999,00  & 4.001--6000    \\
        18  & 2.150,00    & 2.001--4.000   & 43 & 3.900,00  & 2.001--4.000   \\
        19  & 4.595,00    & 4.001--6000    & 44 & 7.000,00  & 6.001--8000    \\
        20  & 5.561,00    & 4.001--6000    & 45 & 3.508,00  & 2.001--4.000   \\
        21  & 2.800,00    & 2.001--4.000   & 46 & 1.130,00  & 0--2.000       \\
        22  & 9.538,00    & 10.001--12.000 & 47 & 4.121,00  & 4.001--6000    \\
        23  & 2.000,00    & 0--2.000       & 48 & 2.601,00  & 2.001--4.000   \\
        24  & 3.226,00    & 2.001--4.000   & 49 & 2.901,00  & 2.001--4.000   \\
        25  & 1.900,00    & 0--2.000       & 50 & 4.871,00  & 4.001--6000    \\
        \bottomrule
    \end{tabular}
    \caption{Renda e faixa de renda dos correntistas}
    \label{tab:ex_01a}
\end{table}
\begin{table}[H]
    \centering
    \begin{tabular}{lcr}
        \toprule
        Faixa de renda & Frequência absoluta & Frequência relativa \\
        \midrule
        0-2.000        & 9                   & 18\% \\
        2.001-4.000    & 19                  & 38\% \\
        4.001-6.000    & 11                  & 22\% \\
        6.001-8.000    & 5                   & 10\% \\
        8.001-10.000   & 4                   & 8\% \\
        10.001-12.000  & 2                   & 4\% \\
        \midrule
        \textbf{Total} & \textbf{50}         & \textbf{100}\% \\
        \bottomrule
    \end{tabular}
    \caption{Tabela de frequências para as faixas de renda}
    \label{tab:ex_01b}
\end{table}
\end{resolucao}

\begin{exercise}
Um analista do mercado acionário coletou os retornos mensais de duas
ações que pretende indicar aos seus clientes. Calcule as estatísticas
descritivas para as duas variáveis, incluindo o coeficiente de correlação entre
os retornos. O banco de dados com os retornos percentuais mensais está na
planilha Lista de Exercício Complementares: aba Exercício 2
--- \Cref{tab:ex_02_origem}.
\begin{table}[htpb]
    \centering
    \begin{tabular}{lcr}
        \toprule
        Meses       & Ação 1  & Ação 2 \\
        \midrule
        1           & -0,0212 & 0,2645 \\
        2           & 0,2438  & 0,2086 \\
        3           & 0,2296  & 0,1248 \\
        4           & -0,2018 & 0,0209 \\
        5           & 0,1296  & 0,2055 \\
        6           & 0,0615  & 0,6260 \\
        7           & -0,1591 & -0,1490 \\
        8           & -0,1001 & 0,2580 \\
        9           & -0,0265 & 0,1722 \\
        10          & 0,0776  & 0,0199 \\
        11          & 0,0370  & 0,4331 \\
        12          & 0,1116  & 0,5482 \\
        13          & -0,0667 & 0,0452 \\
        14          & -0,0082 & -0,1410 \\
        15          & 0,0119  & -0,1059 \\
        16          & 0,1205  & 0,4074 \\
        17          & 0,0477  & -0,0056 \\
        18          & 0,2814  & 0,1482 \\
        19          & -0,0674 & 0,0753 \\
        20          & 0,0762  & 0,0899 \\
        21          & -0,1111 & 0,0160 \\
        22          & -0,0557 & 0,1805 \\
        23          & 0,1991  & 0,0334 \\
        \bottomrule
    \end{tabular}
    \caption{Retornos mensais de duas ações}
    \label{tab:ex_02_origem}
\end{table}
\end{exercise}

\begin{resolucao}
No contexto dessa atividade, temos as seguintes estatísticas descritivas a
serem determinadas:

\paragraph{Média}
Chamando de $X$ o conjunto dos $x$ retornos mensais da ``Ação 1'' e $Y$ o
conjunto dos $y$ retornos mensais da ``Ação 2'', pode-se usar a
\Cref{eq:media} para mostrar que os valores médios $\overline{x}$ e
$\overline{y}$ são iguais a:
\begin{align*}
    \overline{x} &= \frac{-0.0212 + 0.2438 + 0.2296 + \ldots -0.1111 - 0.0557 + 0.1991}{23} \\
                 &= \frac{0.8097}{23} \\
                 &= 0.0352
\end{align*}
\begin{align*}
    \overline{y} &= \frac{0.2645 + 0.2086 + 0.1248 \ldots + 0.0160 + 0.1805 + 0.0334}{23} \\
                 &= \frac{3.4761}{23} \\
                 &= 0.1551
\end{align*}

\paragraph{Mediana}
Ordenando os elementos de $X$, temos que:
\begin{align*}
    X = [& -0.2018, -0.1591, -0.1111, -0.1001, -0.0674, -0.0667, \\
         & -0.0557, -0.0265, -0.0212, -0.0082, 0.0119, \\
         & 0.0370, \\
         & 0.0477, 0.0615, 0.0762, 0.0776, 0.1116, \\
         & 0.1205, 0.1296, 0.1991, 0.2296, 0.2438, 0.2814 ]
\end{align*}
Ordenando os elementos de $Y$, temos que:
\begin{align*}
    Y = [& -0.1490, -0.1410, -0.1059, -0.0056, 0.0160, 0.0199, \\
         & 0.0209, 0.0334, 0.0452, 0.0753, 0.0899, \\
         & 0.1248, \\
         & 0.1482, 0.1722, 0.1805, 0.2055, 0.2086, \\
         & 0.2580, 0.2645, 0.4074, 0.4331, 0.5482, 0.6260 ]
\end{align*}
É possível observar que o termo central e, portanto, a mediana de $X$, é
igual a $0.0370$. De forma análoga, o termo central de $Y$, sua mediana, é
dado por $0.1248$.

\paragraph{Moda} É possível observar que todos os elementos dos conjuntos
$X$ e $Y$ possuem a mesma frequência---logo, não há moda.

\paragraph{Quartis} Usando a \Cref{eq:pos-percentil}, determinaremos, para
as variáveis $x$ e $y$ --- referente aos valores de ``Ação 1'' e
``Ação 2'', respectivamente --- as posições do segundo e do terceiro
quartis --- respectivamente iguais a $p_{25}$ e $p_{75}$ --- os conjuntos
possuem o mesmo número de observações; logo, os percentis encontram-se nas
mesmas posições.
\begin{align*}
    P(p_{25}) &= \left[(23-1) \cdot \left(\frac{25}{100}\right)\right] + 1 \\
    &= 22 \cdot 0.25 + 1 \\
    &= 6.5
\end{align*}
\begin{align*}
    P(p_{75}) &= \left[(23-1) \cdot \left(\frac{75}{100}\right)\right] + 1 \\
    &= 22 \cdot 0.75 + 1 \\
    &= 17.5
\end{align*}
Nota-se que o primeiro quartil está localizado entre as observações 6 e 7
do rol. No cálculo da mediana, ordenamos as observações, de tal modo que
pode-se determinar esses valores como sendo, respectivamente, iguais a
$-0.0667$ e $-0.0557$, para a variável $x$, e $0.0199$ e $0.0209$, para a
variável $y$.
Realizando a interpolação desses dados --- como indicado na
\Cref{subsub:percentis} --- tem-se que:
\begin{align*}
    p_{x25} &=  (-0.0667 \cdot 0.5) + (-0.0557 \cdot 0.5) \\
            &= -0.0334 - 0.0279 \\
            &= -0.0612
\end{align*}
\begin{align*}
    p_{y25} &=  (0.0199 \cdot 0.5) + (0.0209 \cdot 0.5) \\
            &= 0.0010 + 0.0105 \\
            &= 0.0204
\end{align*}
O terceiro quartil está localizado entre as observações 17 e 18.
Interpolando-se esses dados para ambas as variáveis, temos que:
\begin{align*}
    p_{x75} &=  (0.1116 \cdot 0.5) + (0.1205 \cdot 0.5) \\
            &= 0.0558 + 0.0603 \\
            &= 0.1161
\end{align*}
\begin{align*}
    p_{y75} &=  (0.2086 \cdot 0.5) + (0.2580 \cdot 0.5) \\
            &= 0.1043 + 0.1290 \\
            &= 0.2333
\end{align*}

\paragraph{Amplitude} As amplitudes $A_x$ e $A_y$ são dadas por:
\[
    A_x = x_{\textrm{máx}} - x_{\textrm{mín}} =
    0.2814 - (-0.2018) =
    0.4832
\]
\[
    A_y = y_{\textrm{máx}} - y_{\textrm{mín}} =
    0.6260 - (-0.1490) =
    0.7750
\]

\paragraph{Desvio médio} Usando a \Cref{eq:desvio-medio}, podemos
determinar os desvios médios $D_{mx}$ e $D_{my}$:
\begin{align*}
    D_{mx} &=
        \frac{1}{n} \cdot \sum_{i=1}^{n} \lvert x_i - \overline{x} \rvert =
        \frac{1}{23} \cdot \Big( \\
    & \lvert -0.0212 - 0.0352 \rvert + \lvert 0.2438 - 0.0352 \rvert + \lvert 0.2296 - 0.0352 \rvert \\
    & + \ldots + \\
    & \lvert -0.1111 - 0.0352 \rvert + \lvert -0.0557 - 0.0352 \rvert + \lvert 0.1991 - 0.0352 \rvert \\
    & \Big) \implies \\
    D_{mx} &= \frac{2.3863}{23} \implies \\
    D_{mx} &= 0.1038
\end{align*}
\begin{align*}
    D_{my} &=
        \frac{1}{n} \cdot \sum_{i=1}^{n} \lvert y_i - \overline{y} \rvert =
        \frac{1}{23} \cdot \Big( \\
    & \lvert 0.2645 - 0.1551 \rvert + \lvert 0.2086 - 0.1551 \rvert + \lvert 0.1248 - 0.1551 \rvert \\
    & + \ldots + \\
    & \lvert 0.0160 - 0.1551 \rvert + \lvert -0.1805 - 0.1551 \rvert + \lvert 0.0334 - 0.1551 \rvert \\
    & \Big) \implies \\
    D_{my} &= \frac{3.5853}{23} \implies \\
    D_{my} &= 0.1559
\end{align*}

\paragraph{Variância} Utilizando a \Cref{eq:variancia}, determinaremos
as variâncias ${\sigma_x}^2$ e ${\sigma_y}^2$
para as variáveis $x$ e $y$, respectivamente.
\begin{align*}
    {\sigma_x}^2 &=
        \frac{1}{(n-1)} \cdot \sum_{i=1}^{n} (x_i - \overline{x})^2 =
        \frac{1}{22} \cdot \Big( \\
    & ( -0.0212 - 0.0352 )^2 + ( 0.2438 - 0.0352 )^2 + ( 0.2296 - 0.0352 )^2 \\
    & + \ldots + \\
    & ( -0.1111 - 0.0352 )^2 + ( -0.0557 - 0.0352 )^2 + ( 0.1991 - 0.0352)^2 \\
    & \Big) \implies \\
    {\sigma_x}^2 &= \frac{0.3674}{22} \implies \\
    {\sigma_x}^2 &= 0.0167
\end{align*}
\begin{align*}
    {\sigma_y}^2 &=
        \frac{1}{(n-1)} \cdot \sum_{i=1}^{n} (y_i - \overline{y})^2 =
        \frac{1}{22} \cdot \Big( \\
    & ( 0.2645 - 0.1551 )^2 + ( 0.2086 - 0.1551 )^2 + ( 0.1248 - 0.1551 )^2 \\
    & + \ldots + \\
    & ( 0.0160 - 0.1551 )^2 + ( -0.1805 - 0.1551 )^2 + ( 0.0334 - 0.1551 )^2 \\
    & \Big) \implies \\
    {\sigma_y}^2 &= \frac{0.9140}{22} \implies \\
    {\sigma_y}^2 &= 0.0415
\end{align*}

\paragraph{Desvio-padrão} Os desvios-padrões $\sigma_x$ e $\sigma_y$ são
determinados a partir da \Cref{eq:desvio-padrao}:
\[
    \sigma_x = \sqrt{{\sigma_x}^2} = \sqrt{0.0167} = 0.1292
\]
\[
    \sigma_y = \sqrt{{\sigma_y}^2} = \sqrt{0.0415} = 0.2037
\]

\paragraph{Erro padrão} Usando a \Cref{eq:erro-padrao}, tem-se que os erros
padrões $\sigma_{\overline{x}}$ e $\sigma_{\overline{y}}$ são dados por:
\[
\sigma_{\overline{x}} = \frac{\sigma_x}{\sqrt{n}} =
    \frac{0.1292}{\sqrt{23}} = 0.0269
\]
\[
\sigma_{\overline{y}} = \frac{\sigma_y}{\sqrt{n}} =
    \frac{0.2038}{\sqrt{23}} = 0.0425
\]

\paragraph{Coeficiente de variação} Os coeficientes de variação $CV_x$ e
$CV_y$ podem ser determinados pela \Cref{eq:coef-variacao}:
 \[
     CV_x = \frac{\sigma_x}{\overline{x}}
     = \frac{0.0167}{0.0352} = 0.4743
\]
 \[
     CV_y = \frac{\sigma_y}{\overline{y}}
     = \frac{0.0415}{0.1511} = 0.2749
\]

\paragraph{Resumo} A tabela \ref{tab:resumo-m02ex02} resume as estatísticas
descritivas para as variáveis ``Ação 1'' e ``Ação 2''.
\begin{table}[htpb]
    \centering
    \begin{tabular}{lcc}
    \toprule
                            & Ação 1 ($x$) & Ação 2 ($y$) \\
    \midrule
    Média                   & 0.0352       & 0.1511 \\
    Mediana                 & 0,0370       & 0,1248 \\
    Moda                    & não há       & não há \\
    Primeiro quartil        & -0,0612      & 0,0204 \\
    Terceiro quartil        & 0,1161       & 0,2333 \\
    Amplitude               & 0,4832       & 0,775 \\
    Desvio médio            & 0,1038       & 0,1559 \\
    Variância               & 0,0167       & 0,0415 \\
    Desvio-padrão           & 0,1292       & 0,2038 \\
    Erro padrão             & 0,0269       & 0,0425 \\
    Coeficiente de variação & 0,4743       & 0,2749 \\
    \bottomrule
    \end{tabular}
    \caption{Estatísticas descritivas para as variáveis ``Ação 1'' e ``Ação 2''}
    \label{tab:resumo-m02ex02}
\end{table}
\end{resolucao}

\begin{exercise}
Em certo jogo, probabilidade de vitória (sucesso) a cada nova jogada é 1/6.
Se forem feitas 10 jogadas, quais são as seguintes probabilidades:
\begin{enumerate}[label=\alph*)]
    \item Ter vitória em 4 jogadas.
    \item Ter vitória em pelo menos 7 jogadas.
\end{enumerate}
\end{exercise}

\begin{resolucao}
No aguardo ...
\end{resolucao}
