\subsection{Distribuições para variáveis aletórias contínuas}%
\lesson{3}{Sex 18 out 2023 19:00}{Distribuições para variáveis aletórias contínuas}

\subsubsection{Distribuição uniforme}

Anotar depois, conferir \parencite[p.~327]{favero}.

\subsubsection{Distribuição normal}

Também conhecida como \textbf{distribuição Gaussiana}, é a mais utilizada e
importante dentre todas as distribuições, pois permite modelar diversos
fenômenos naturais, econômicos e sociais.

A distribuição normal é frequentemente usada para modelar variáveis que se
distribuem de maneira próxima ao padrão médio e simétrico. Alguns exemplos
incluem:

\begin{enumerate}
\item Altura e peso de uma população: as alturas e pesos de indivíduos de uma
população tendem a seguir uma distribuição normal.
\item Erros de medição: os erros de medição em experimentos, como os pequenos
desvios entre medições reais e valores teóricos, costumam se distribuir
normalmente.
\item Desempenho de estudantes em testes: em um grande grupo, as notas de
estudantes em testes padronizados geralmente seguem uma distribuição
normal, com a maioria concentrada em torno da média.
\item Flutuações no mercado financeiro: em algumas condições, pequenas
variações diárias nos preços de ações e índices financeiros podem ser
modeladas como uma distribuição normal.
\item Processos biológicos e físicos: características biológicas, como a
pressão arterial, e medições físicas, como a intensidade do som em um
ambiente, frequentemente seguem uma distribuição normal devido a sua
variabilidade natural.
\end{enumerate}

Seja $X$ uma variável aleatória que segue uma distribuição normal, com média
$\mu \in \R$ e desvio-padrão $\sigma > 0$. Sua função de densidade de
probabilidade $f(x)$ é dada por
\begin{equation}
    f(x) = \frac{1}{\sigma \cdot \sqrt{2 \pi}} \cdot e^{-\frac{(x-\mu)^2}{2 \sigma^2}}
    \label{eq:dist-normal}
\end{equation}
com $-\infty < x < +\infty$.

\begin{figure}[htpb]
    \centering
    {\input{graphics/dist_normal1.pdf_tex}}
    \caption{Distribuição normal}
    \label{fig:dist_normal1}
\end{figure}

\begin{figure}[htpb]
    \centering
    {\input{graphics/dist_normal2.pdf_tex}}
    \caption{Frequência de probabilidades em uma distribuição normal}
    \label{fig:dist_normal2}
\end{figure}

A figura \Cref{fig:dist_normal1} ilustra graficamente uma distribuição normal,
a partir da qual pode-se notar sua curva característica em formato de sino, com
simetria em torno da média. Na figura \Cref{fig:dist_normal2}, observa-se outra
propriedade particular e importante dessa distribuição: aproximadamente 68\%
dos valores estão situados 1 desvio-padrão acima ou abaixo da média, enquanto
95\% situam-se em até 2 desvios-padrões da média e, por fim, quase 99,7\% dos
valores concentram-se na região de 3 desvios-padrões em torno da média. Também
pode-se observar que, quanto menor o desvio-padrão $\sigma$, mais concentrada é
a curva em torno da média $\mu$.

Também da \Cref{fig:dist_normal2}, nota-se que o valor esperado é $E(X)=\mu$,
de tal forma que também pode-se mostrar que a variância é dada por
$Var(X)=\sigma^2$.

Contudo, é mais conveniente e comum trabalhar com a \textbf{distribuição normal
padrão} --- ou distrbuição normal reduzida. Ta distribuição é obtida a partir
da transformação da variável $X$ em uma nova variável aleatória $Z$, conhecida
como \emph{z-score} (escore padrão), com média $\mu=0$ e variância
$\sigma^2=1$, determinada por:
\begin{equation}
    Z = \frac{X - \mu}{\sigma}
    \label{eq:zscore}
\end{equation}

Interpretamos a equação \Cref{eq:zscore} da seguinte maneira:

\begin{itemize}
    \item Um \emph{z-score} igual a 0 indica que o valor $x$ da variável
    correspondente é exatamente igual à média --- afinal, tem-se neste caso que
    $x=\mu$, de posse que $x-\mu = 0$.
    \item Um \emph{z-score} positivo indica que o valor $x$ da variável
    correspondente está acima da média. Mais especificamente, para $Z=1$
    tem-se que $x$ está 1 desvio-padrão acima da média; para $Z=2$, $x$ está 2
    desvios-padrões acima da média, e assim sucessivamente.
    \item Um \emph{z-score} negativo indica que o valor $x$ da variável
    correspondente está abaixo da média. Mais especificamente, para $Z=-1$
    tem-se que $x$ está 1 desvio-padrão abaixo da média; para $Z=-2$, $x$ está
    2 desvios-padrões abaixo da média, e assim sucessivamente.
\end{itemize}

De posse do que foi discutido anteriormente --- e conforme apresentado na
\Cref{fig:dist_normal2} --- podemos dizer também que aproximadamente 68\% dos
valores em uma distribuição normal padrão possuem \emph{z-score} entre $-1$ e
$1$, 95\% possuem \emph{z-score} entre $-2$ e $2$, enquanto 99\% possuem
\emph{z-score} entre $-3$ e $3$.

Esse tipo de transformação permite comparar distribuições de variáveis
diferentes, com distintas métricas ou ordens de grandeza, porque não altera as
formas das distribuições originais e gera novas variáveis com iguais valores de
média e variância. É fácil perceber como pode-se transformar a representação de
uma distribuição normal, dada pela \Cref{fig:dist_normal1}, com a aplicação de
escores padrões --- conferir a \Cref{fig:dist_zscore}.

\begin{figure}[htpb]
    \centering
    {\input{graphics/dist_zscore.pdf_tex}}
    \caption{Distribuição normal padrão}
    \label{fig:dist_zscore}
\end{figure}

A função de distribuição de probabilidade $f(z)$ é dada por:
\begin{equation}
    f(z) = \frac{1}{\sqrt{2 \pi}} \cdot e^{-\nicefrac{z^2}{2}}
    \label{eq:dist-normal-zscore}
\end{equation}

Podemos obter a frequência acumulada em uma distribuição normal a partir da
expressão seguinte:
\begin{equation}
    F(x_c) = P(X \leq x_c) = \int_{{-\infty}}^{{x_c}} {f(x)} \: d{x} {}
    \label{eq:freq-acumulada-dist-normal}
\end{equation}

Matematicamente, a \Cref{eq:freq-acumulada-dist-normal} corresponde ao cálculo
da área sob a curva de $f(x)$, no intervalo $-\infty$ a $x_c$. Interpreta-se
isso como a determinação de todas as probabilidades acumuladas de ocorrências
de valores para a variável aleatória $x$, até um valor de referência $x_c$.

No caso de uma distribuição normal padrão, a frequência acumulada $F(z_c)$ pode
ser obtida por:
\begin{equation}
    F(z_c) = P(Z \leq z_c) = \int_{{-\infty}}^{{z_c}} {f(z)} \: d{z} {} =
    \frac{1}{2 \pi} \cdot \int_{{-\infty}}^{{z_c}} {e^{-\nicefrac{z^2}{2}}} \: d{z} {}
    \label{eq:freq-acumulada-dist-padrao}
\end{equation}

Dado que a \Cref{eq:freq-acumulada-dist-padrao} não tem solução analítica
simples --- pessoalmente não tenho ideia de como \emph{iniciar} a resolução ---
geralmente se recorre a tabelas de \emph{z-score}, como apresentado em
\parencite[p.~331]{favero} e \parencite[p.~2008--2009]{favero}.
