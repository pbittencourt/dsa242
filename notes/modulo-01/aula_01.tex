\section{Estatística descritiva}%
\label{sec:estatistica-descritiva}
\lesson{1}{Sex 11 out 2023 19:00}{Estatística descritiva}

% -----------------------------------------------------------------------------
% Anotações gerais -- remover depois
% Uma prova para o módulo todo, ou seja, disponibilizado ao final da sexta aula
% (3 estatística + 3 python).
%
% O excel da lista complementar não contém apenas as resoluções, parte dos dados
% dos enunciados também está ali. Poderia ter separado. Porque eu preciso abrir a
% planilha para iniciar a resolução e já vejo ali as respostas.
%
% Chuva e alimento. Perdi aproximadamente 1h de aula, entre 20h e 21h. Acho que
% ele ficou só nas tabelas de frequência. Retomei no início das medidas de
% posição. Depois temos que resgatar se algo relevante foi dito e necessita de
% anotação. Meu palpite é que nada foi perdido.
% -----------------------------------------------------------------------------

Estrutura tabular de dados é o mais comum de se utilizar. Dimensão linha e
dimensão coluna. Esse cruzamento resulta numa estrutura tabular --- igual a de
uma planilha de excel. As unidades de observação estão nas linhas e os
atributos (variáveis) a serem medidos ou classificados estão nas colunas.

O ponto de partida de uma análise de dados é compreender as características dos
dados que estão sendo analisados. Isso implica em entender os tipos de
variáveis envolvidas. Essa distinção é fundamental porque vai determinar a
escolha da técnica a ser utilizada.

Essa aula tem como foco a \textbf{estatística descritiva}, dada por meio de
\emph{medidas resumo}. O objetivo principal dessas medidas é representar o
comportamento da variável em estudo por meio de seus valores centrais e não
centrais e suas dispersões ou formas de distribuição dos seus valores em torno
da média.
Estudaremos medidas de posição ou localização (medidas de tendência central e
medidas separatrizes), medidas de dispersão ou variabilidade e medidas de
forma, como assimetria e curtose \parencite[p.~112]{favero}.

Ao longo da \Cref{sec:estatistica-descritiva}, estudaremos o seguinte caso:

\begin{case}
    Realizou-se uma coleta de preços de um determinado produto em 100
    diferentes locais de venda, ao longo de um certo intervalo de tempo. A
    \Cref{tab:medidas-posicao} contém a tabulação desses dados. Na
    \Cref{tab:medidas-posicao-ordenado}, temos as mesmas informações, ordenadas
    de forma crescente. Determine as estatísticas descritivas para a variável
    preço e construa a tabela resumo.
    \label{eg:coleta-precos}
\end{case}

\begin{table}
    \centering
    \begin{tabular}{cccccccccc}
    \toprule
    Observação & preço (R\$) & O  & p      & O  & p      & O   & p \\
    \midrule
    1          & 189,00      & 26 & 215,00 & 51 & 199,00 & 76  & 185,00 \\
    2          & 195,00      & 27 & 149,00 & 52 & 209,00 & 77  & 179,00 \\
    3          & 199,00      & 28 & 189,00 & 53 & 229,00 & 78  & 169,00 \\
    4          & 189,00      & 29 & 169,00 & 54 & 199,00 & 79  & 179,00 \\
    5          & 197,00      & 30 & 179,00 & 55 & 195,00 & 80  & 189,00 \\
    6          & 189,00      & 31 & 159,00 & 56 & 199,00 & 81  & 199,00 \\
    7          & 199,00      & 32 & 199,00 & 57 & 179,00 & 82  & 209,00 \\
    8          & 202,00      & 33 & 195,00 & 58 & 169,00 & 83  & 169,00 \\
    9          & 199,00      & 34 & 189,00 & 59 & 189,00 & 84  & 159,00 \\
    10         & 209,00      & 35 & 209,00 & 60 & 205,00 & 85  & 179,00 \\
    11         & 189,00      & 36 & 196,00 & 61 & 199,00 & 86  & 185,00 \\
    12         & 179,00      & 37 & 189,00 & 62 & 189,00 & 87  & 189,00 \\
    13         & 175,00      & 38 & 165,00 & 63 & 189,00 & 88  & 179,00 \\
    14         & 199,00      & 39 & 170,00 & 64 & 199,00 & 89  & 199,00 \\
    15         & 205,00      & 40 & 179,00 & 65 & 179,00 & 90  & 199,00 \\
    16         & 219,00      & 41 & 170,00 & 66 & 189,00 & 91  & 189,00 \\
    17         & 229,00      & 42 & 175,00 & 67 & 239,00 & 92  & 169,00 \\
    18         & 205,00      & 43 & 169,00 & 68 & 215,00 & 93  & 159,00 \\
    19         & 190,00      & 44 & 189,00 & 69 & 199,00 & 94  & 169,00 \\
    20         & 179,00      & 45 & 195,00 & 70 & 179,00 & 95  & 209,00 \\
    21         & 199,00      & 46 & 199,00 & 71 & 195,00 & 96  & 189,00 \\
    22         & 189,00      & 47 & 199,00 & 72 & 199,00 & 97  & 179,00 \\
    23         & 183,00      & 48 & 199,00 & 73 & 209,00 & 98  & 189,00 \\
    24         & 199,00      & 49 & 189,00 & 74 & 205,00 & 99  & 199,00 \\
    25         & 206,00      & 50 & 182,00 & 75 & 179,00 & 100 & 195,00 \\
    \bottomrule
    \end{tabular}
    \caption{Preços de um produto coletado em 100 pontos de venda}
    \label{tab:medidas-posicao}
\end{table}

\begin{table}
    \centering
    \begin{tabular}{cccccccccc}
    \toprule
    Posição  & preço (R\$) & P  & p      & P  & p      & P   & p \\
    \midrule
    1        & 149,00      & 26 & 179,00 & 51 & 189,00 & 76  & 199,00 \\
    2        & 159,00      & 27 & 179,00 & 52 & 189,00 & 77  & 199,00 \\
    3        & 159,00      & 28 & 179,00 & 53 & 190,00 & 78  & 199,00 \\
    4        & 159,00      & 29 & 179,00 & 54 & 195,00 & 79  & 199,00 \\
    5        & 165,00      & 30 & 182,00 & 55 & 195,00 & 80  & 199,00 \\
    6        & 169,00      & 31 & 183,00 & 56 & 195,00 & 81  & 199,00 \\
    7        & 169,00      & 32 & 185,00 & 57 & 195,00 & 82  & 199,00 \\
    8        & 169,00      & 33 & 185,00 & 58 & 195,00 & 83  & 202,00 \\
    9        & 169,00      & 34 & 189,00 & 59 & 195,00 & 84  & 205,00 \\
    10       & 169,00      & 35 & 189,00 & 60 & 196,00 & 85  & 205,00 \\
    11       & 169,00      & 36 & 189,00 & 61 & 197,00 & 86  & 205,00 \\
    12       & 169,00      & 37 & 189,00 & 62 & 199,00 & 87  & 205,00 \\
    13       & 170,00      & 38 & 189,00 & 63 & 199,00 & 88  & 206,00 \\
    14       & 170,00      & 39 & 189,00 & 64 & 199,00 & 89  & 209,00 \\
    15       & 175,00      & 40 & 189,00 & 65 & 199,00 & 90  & 209,00 \\
    16       & 175,00      & 41 & 189,00 & 66 & 199,00 & 91  & 209,00 \\
    17       & 179,00      & 42 & 189,00 & 67 & 199,00 & 92  & 209,00 \\
    18       & 179,00      & 43 & 189,00 & 68 & 199,00 & 93  & 209,00 \\
    19       & 179,00      & 44 & 189,00 & 69 & 199,00 & 94  & 209,00 \\
    20       & 179,00      & 45 & 189,00 & 70 & 199,00 & 95  & 215,00 \\
    21       & 179,00      & 46 & 189,00 & 71 & 199,00 & 96  & 215,00 \\
    22       & 179,00      & 47 & 189,00 & 72 & 199,00 & 97  & 219,00 \\
    23       & 179,00      & 48 & 189,00 & 73 & 199,00 & 98  & 229,00 \\
    24       & 179,00      & 49 & 189,00 & 74 & 199,00 & 99  & 229,00 \\
    25       & 179,00      & 50 & 189,00 & 75 & 199,00 & 100 & 239,00 \\
    \bottomrule
    \end{tabular}
    \caption{Valores de \Cref{tab:medidas-posicao}, ordenados de modo crescente}
    \label{tab:medidas-posicao-ordenado}
\end{table}

\subsection{Medidas de posição}%

\subsubsection{Média}
A média aritmética simples $\overline{x}$ dos $x$ elementos de um conjunto é
dada pela expressão
\begin{equation}
    \overline{x} = \frac{1}{n} \cdot \sum_{i=1}^{n} x_i
    \label{eq:media}
\end{equation}
em que $n$ é a quantidade de elementos do conjunto e $x_i$ representa cada
um desses valores. No \Cref{eg:coleta-precos}, tem-se que a média das
observações é dada por
\begin{align*}
    \overline{x} &= \frac{189+195+199 + \ldots + 189+199+195 }{100} \\
                 &= \frac{19077}{100} \\
                 &= 190.77
\end{align*}

\subsubsection{Média ponderada}
No caso da média aritmética simples, todas as ocorrências têm o mesmo peso,
isto é, a mesma frequência (ou importância). Se atribuirmos pesos $p_i$ para
cada valor $x_i$ da variável $X$, podemos calcular a média aritmética ponderada
$\overline(x)$, dada por:
\begin{equation}
    \overline{x} = \frac{\sum_{i=1}^{n} x_i \cdot p_i}{\sum_{i=1}^{n} p_i}
    \label{eq:media-ponderada}
\end{equation}
Se o peso estiver expresso de forma relativa ao total ($f_i$), podemos
reescrever a \Cref{eq:media-ponderada} como:
\begin{equation}
    \overline{x} = \sum_{i=1}^{n} x_i \cdot f_i
    \label{eq:media-ponderada-relativa}
\end{equation}
Todo mundo que passou pela escola já lidou com isso. Imaginemos que, no ensino
médio, a nota bimestral era composta por três avaliações, todas valendo entre 0
e 10, mas com diferentes importâncias:
\begin{itemize}
    \item mensal, de peso 2;
    \item bimestral, de peso 2;
    \item contínua, de peso 1.
\end{itemize}
Um estudante fictício, Pedro, obteve, respectivamente, as seguintes notas: 8, 9
e 7. Utilizando a \Cref{eq:media-ponderada}, temos que sua nota bimestral
$\overline{x}$ é dada por:
 \[
    \overline{x} = \frac{2 \cdot 8 + 2 \cdot 9 + 1 \cdot 7}{2 + 2 + 1}
    = \frac{41}{5} = 8.2
\]
Isso é equivalente a dizer que ele teve duas notas bimestrais, duas mensais e
uma contínua, totalizando cinco notas. Podemos representar os pesos dessas
avaliações como sendo respectivamente iguais a 0.4, 0.4 e 0.2 --- 40\%, 40\% e
20\%. Ou seja, a avaliação mensal, por exemplo, compõe 40\% da sua nota final.
É importante notar que a soma dessas frequências deve sempre ser igual a 1.
Deste modo, utilizando a \Cref{eq:media-ponderada-relativa}, temos que sua nota
bimestral $\overline{x}$ é dada por:
 \[
    \overline{x} = 0.4 \cdot 8 + 0.4 \cdot 9 + 0.2 \cdot 7 = 8.2
\]

\subsubsection{Mediana}
Dispondo-se dos $n$ valores do conjunto em rol, tem-se que a mediana $Md$ é
dada por:
\begin{equation}
    \begin{split}
    Md =
    \begin{cases}
        x_i \textrm{, com } i = \frac{n+1}{2} & \textrm{, se } n \textrm{ for ímpar} \\
        \frac{x_i + x_{i+2}}{2} \textrm{, com } i = \frac{n}{2} & \textrm{, se } n \textrm{ for par}
    \end{cases}
    \end{split}
\end{equation}
De forma menos rigorosa, podemos definir a determinação da mediana da
seguinte maneira:
\begin{enumerate}
    \item Ordena-se todos os elementos do conjunto, de forma crescente
    --- a este ordenamento chamamos de \emph{rol}.
    \item Se o conjunto tiver quantidade \emph{ímpar} de elementos, a
    mediana corresponde ao termo que está no centro da distribuição.
    \item Se a quantidade for \emph{par}, dividimos o conjunto ao meio e
    tomamos os dois elementos que determinam essa divisão --- o último do
    grupo à esquerda e o primeiro do grupo à direita. A média aritmética
    simples entre esses elementos corresponderá à mediana.
\end{enumerate}

Para o \Cref{eg:coleta-precos}, o número de observações é par. A mediana $Md$
é dada pela média entre os 50º e 51º elementos:
\[
Md=\frac{189+189}{2}=189
\]

\subsubsection{Moda}
A moda de um conjunto de dados corresponde ao elemento que ocorre com a maior
frequência --- a moda do conjunto $[1,2,2,3,5]$ é $2$, por exemplo.

Analisando os dados do \Cref{eg:coleta-precos}, a partir da
\Cref{tab:medidas-posicao-ordenado}, pode-se mostrar que a observação mais
frequente e, portanto, a moda do conjunto, é dada por 199.

\subsubsection{Percentis}
\label{subsub:percentis}

A posição $P$ do i-ésimo percentil $p_i$ de uma distribuição é determinada por:
\begin{equation}
    P(p_i)=\left[(n-1) \cdot \left(\frac{i}{100}\right)\right] + 1
    \label{eq:pos-percentil}
\end{equation}
Vamos determinar a posição do 25º percentil $p_{25}$ da amostra indicada pela
\Cref{tab:medidas-posicao-ordenado} --- note que o 25º percentil corresponde
também ao primeiro quartil da amostra.
\begin{align*}
    P(p_{25})&=\left[(n-1) \cdot \left(\frac{i}{100}\right)\right] + 1 \\
             &=\left[(100-1) \cdot \left(\frac{25}{100}\right)\right] + 1 \\
             &=99 \cdot 0.25 + 1 \\
             &=25.75
\end{align*}
Isso indica que a observação que determina o primeiro quartil está entre as
posições 25 e 26. De acordo com a \Cref{tab:medidas-posicao-ordenado},
observa-se que em ambas as posições as observações possuem o mesmo valor
e, portanto, este corresponde ao primeiro quartil. Ou seja:
\[
    p_{25}=179
\]

Como interpretamos essa informação: podemos dizer que 25\% dos produtos custam
R\$ 179,00 ou menos, enquanto os outros 75\% custam mais do que R\$ 179,00.

Nessa amostra, dado que as observações 25 e 26 possuem o mesmo valor, não é
necessário nenhum passo adicional para determinar o percentil.

Quando esses valores forem diferentes, precisamos realizar a
\emph{interpolação} entre eles. Utilizando a \Cref{eq:pos-percentil},
pode-se mostrar que $P(p_{60})=60.4$. Da
\Cref{tab:medidas-posicao-ordenado}, temos que os elementos das posições 60 e 61
são, respectivamente, iguais a 196 e 197. Devemos encontrar o elemento que
está, a partir de 196, a 40\% da distância entre 196 e 197 --- analogamente,
ele também está a 60\% dessa distância. Matematicamente:

\begin{align*}
    p_{60} &= 196 + 0.4 \cdot (197-196) \\
           &= 196 + 197 \cdot 0.4 - 196 \cdot 0.4 \\
           &= 196 \cdot 0.6 + 197 \cdot 0.4 \\
           &= 196.4
\end{align*}

\subsection{Medidas de dispersão}%
\label{subsub:medidas-dispersao}

\subsubsection{Amplitude}

É a medida mais simples de dispersão. A amplitude $A$ é dada pela diferença
entre os valores máximo e mínimo do conjunto de observações:

\begin{equation}
    A = x_{\textrm{máx}} - x_{\textrm{mín}}
    \label{eq:amplitude}
\end{equation}

Para o caso do \Cref{eg:coleta-precos} a amplitude $A$ é dada por:
\[
A=239.00-149.00=90
\]

\subsubsection{Desvio médio}

O desvio $D_i$ de um elemento é a diferença entre cada valor do conjunto e a
média aritmética deste conjunto. Matematicamente:
\[
    D_i = x_i - \overline{x}
\]
O desvio médio $D_m$ é a média dos desvios \textbf{absolutos} de cada elemento,
isto é, tratados em módulo:
\begin{equation}
    D_m = \frac{1}{n} \cdot \sum_{i=1}^{n} \lvert x_i - \overline{x} \rvert
    \label{eq:desvio-medio}
\end{equation}
Em relação ao \Cref{eg:coleta-precos}, o desvio médio $D_m$ é dado por:
\begin{align*}
    D_m &= \frac{1}{100} \cdot \Big( \\
    &\lvert 189.00-190.77 \rvert + \lvert 195.00-190.77 \rvert + \lvert 199.00-190.77 \rvert + \\
    &+ \ldots + \\
    &\lvert 189.00-190.77 \rvert + \lvert 199.00-190.77 \rvert + \lvert 195.00-190.77 \rvert \\
    \Big) &\implies \\
    D_m &= \frac{1.77 + 4.23 + 8.23 + \ldots + 1.77 + 8.23 + 4.23}{100} \implies \\
    D_m &= \frac{1207.62}{100} \therefore \\
    D_m &= 12.08
\end{align*}

\subsubsection{Variância}

Dispersão das observações de uma variável em torno de sua média. A variância
$\sigma^2$ pode ser determinada pela expressão
\begin{equation}
    \sigma^2 = \frac{1}{(n-1)} \cdot \sum_{i=1}^{n} (x_i - \overline{x})^2
    \label{eq:variancia}
\end{equation}
em que $n$ representa o tamanho da amostra, $x_i$ representa o i-ésimo elemento
da amostra e $\overline{x}$ indica a média aritmética simples das observações.

Usando dados da \Cref{tab:medidas-posicao}, vemos que a variância das
observações no \Cref{eg:coleta-precos} é determinada por

\begin{align*}
    \sigma^2 &= \frac{1}{99} \cdot \Big( \\
    &(189.00-190.77)^2 + (195.00-190.77)^2 + (199.00-190.77)^2 + \\
    &+ \ldots + \\
    &(189.00-190.77)^2 + (199.00-190.77)^2 + (195.00-190.77)^2 \\
    \Big) &\implies \\
    \sigma^2 &= \frac{3.13 + 17.89 + 67.73 + \ldots + 3.13 + 67.73 + 17.89}{99} \implies \\
    \sigma^2 &= \frac{24157.71}{99} \therefore \\
    \sigma^2 &= 244.02
\end{align*}

\subsubsection{Desvio-padrão}

Interpretar a dispersão a partir da variância é pouco viável, dado que a
dimensão está elevada ao quadrado. Utiliza-se, então, o desvio-padrão em
relação à amostra, determinado pela expressão
\begin{equation}
    \sigma = \sqrt{\sigma^2}
    \label{eq:desvio-padrao}
\end{equation}
No \Cref{eg:coleta-precos}, o desvio-padrão em relação à amostra é dado por
\[
\sigma=\sqrt{244.02}=15.621
\]
Isso sugere que a maior parte dos valores da amostra está distribuída dentro de
um intervalo de aproximadamente R\$ 15.62 reais acima ou abaixo da média. Ou
seja, grande parte dos dados está entre $190.77-15.62=175.15$ e
$190.77+15.62=206.39$.

\subsubsection{Erro padrão}

Corresponde ao desvio-padrão da média e pode ser obtido pela expressão
\begin{equation}
    \sigma_{\overline{x}} = \frac{\sigma}{\sqrt{n}}
    \label{eq:erro-padrao}
\end{equation}
Para o \Cref{eg:coleta-precos}, tem-se que o erro padrão é dado por
\[
    \sigma_{\overline{x}}=\frac{15.621}{\sqrt{100}}=1.562
\]

\subsubsection{Coeficiente de variação}

O coeficiente de variação $CV$ é uma medida de dispersão relativa que fornece a
variação dos dados em relação à sua média.
\begin{equation}
    CV = \frac{\sigma}{\overline{x}}
    \label{eq:coef-variacao}
\end{equation}
Quanto menor o coeficiente de variação, mais homogêneos são os dados, isto é,
menor a dispersão em torno da média. Um CV pode ser considerado baixo,
indicando um conjunto de dados razoavelmente homogêneo, quando for menor do que
30\%. Se esse valor for acima de 30\%, o conjunto de dados pode ser considerado
heterogêneo \parencite[p.~159]{favero}. Em relação ao \Cref{eg:coleta-precos},
tem-se que o coeficiente de variação é dado por
\[
CV = \frac{15.621}{190.77} = 0.08188 = 8.19\%
\]
Observa-se que os dados são homogêneos, indicando que a média é uma boa medida
para representá-los.

\subsection{Medidas de forma}%

Estudaremos medidas de assimetria (\emph{skewness}) e curtose
(\emph{kurtosis}), que caracterizam como os elementos do conjunto estão
distribuídos em torno da média.

\subsubsection{Medidas de assimetria}

Uma distribuição é dita \textbf{simétrica} quando sua média, mediana e moda
coincidem. A curva de distribuição é bem representada pela
\Cref{fig:dist-simetrica}; o eixo das abscissas representa o valor da
observação, enquanto o eixo das ordenadas indica a frequência desse valor na
distribuição.

\begin{figure}[htpb]
    \centering
    \input{graphics/dist_simetrica.pdf_tex}
    \caption{Representação de uma distribuição simétrica}
    \label{fig:dist-simetrica}
\end{figure}

Se a distribuição se concentrar do lado esquerdo, criando uma ``cauda''
alongada no lado direito, temos uma \textbf{distribuição assimétrica positiva}
ou \textbf{distribuição assimétrica à direita}---\Cref{fig:dist-direita}.
Estando a distribuição concentrada do lado direito, portanto com a ``cauda''
alongada no lado esquerdo, temos uma \textbf{distribuição assimétrica negativa}
ou \textbf{distribuição assimétrica à esquerda}---\Cref{fig:dist-esquerda}.

\begin{figure}[ht!]
    \centering
    \begin{subfigure}{0.48\textwidth}
        \centering
        % \resizebox{\textwidth}{!}{\input{image1.pdf}}
        \input{graphics/dist_direita.pdf_tex}
        \caption{Distribuição assimétrica à direita}
        \label{fig:dist-direita}
    \end{subfigure}
    \hfill
    \begin{subfigure}{0.48\textwidth}
        \centering
        \input{graphics/dist_esquerda.pdf_tex}
        \caption{Distribuição assimétrica à esquerda}
        \label{fig:dist-esquerda}
    \end{subfigure}
    \caption{Representação de distribuições assimétricas}
    \label{fig:dist-assimetrica}
\end{figure}

Vejamos maneiras de determinar os coeficientes de assimetria.

\paragraph{Primeiro coeficiente de assimetria de Pearson}
Medida de assimetria dada pela diferença entre a média e a moda, ponderada pelo
desvio-padrão \parencite[p.~161]{favero}. Matematicamente, pode-se expressar o
primeiro coeficiente de assimetria de Pearson $A_{s_1}$ como
\begin{equation}
    A_{s_1} = \frac{\overline{x} - Mo}{\sigma}
    \label{eq:prim-coef-ass-pearson}
\end{equation}
em que:
\begin{itemize}
    \item se $A_{s_1}$ é igual a $0$, a distribuição é simétrica.
    \item se $A_{s_1}$ é maior do que $0$, a distribuição é assimétrica à
    direita --- distribuição positiva.
    \item se $A_{s_1}$ é menor do que $0$, a distribuição é assimétrica à
    esquerda --- distribuição negativa.
\end{itemize}

Retomando o \Cref{eg:coleta-precos}, determinemos $A_{s_1}$:
\[
    A_{s_1} = \frac{190.77 - 199}{15.621} = \frac{-8.23}{15.621} = -0.527
\]
Observa-se que $A_{s_1}<0$, indicando uma distribuição assimétrica à
esquerda---representada pela \Cref{fig:dist-esquerda}. Na imagem, pode-se notar
também que, para esse tipo de distribuição, $\overline{x}<Md<Mo$. Entretanto
\footnote{Precisamos entender se isso indica um erro em nossos cálculos, alguma
imprecisão da definição fornecida pelo professor ou se a relação apontada nem
sempre é verdadeira.},
na situação em estudo, nota-se que $Md<\overline{x}<Mo$

\paragraph{Segundo coeficiente de assimetria de Pearson}
Medida de assimetria que não é dada em função da moda da distribuição. O
segundo coeficiente de assimetria de Pearson $A_{s_2}$ é determinado por:
\begin{equation}
    A_{s_2} = \frac{3 \cdot (\overline{x} - Md)}{\sigma}
    \label{eq:seg-coef-ass-pearson}
\end{equation}
As relações de assimetria são iguais para segundo e primeiro coeficientes, isto
é:
\begin{itemize}
    \item se $A_{s_2}=0$, a distribuição é simétrica.
    \item se $A_{s_2}>0$, a distribuição é positiva.
    \item se $A_{s_2}<0$, a distribuição é negativa.
\end{itemize}
Determinando o segundo coeficiente de assimetria de Pearson para o
\Cref{eg:coleta-precos}, temos:
\[
    A_{s_2} = \frac{3 \cdot (190.77-189)}{15.621} = \frac{5.31}{15.621} = 0.340
\]
Podemos notar que $A_{s_2}>0$, denotando uma assimetria positiva, isto é, à
direita---representada pela \Cref{fig:dist-direita}. Temos um conflito com o
cálculo do primeiro coeficiente, que indicada assimetria em sentido contrário.
Acredito que o segundo seja mais confiável --- na apresentação de slides, o
professor sugere a relação apenas entre média e mediana, desconsiderando a
moda. Isto é:
\begin{itemize}
    \item quando $\overline{x}$ > $Md$, a distribuição é assimétrica à direita
    (positiva);
    \item quando $\overline{x}$ < $Md$, a distribuição é assimétrica à esquerda
    (negativa).
\end{itemize}

\paragraph{Coeficiente de assimetria de Bowley}
Também conhecido como \textbf{coeficiente quartílico de assimetria}, o
coeficiente de assimetria de Bowley $A_{s_B}$ é dado em função das medidas
separatrizes---primeiro, segundo e terceiro quartis:
\begin{equation}
    A_{s_B} = \frac{Q_3+Q_1-2 \cdot Q_2}{Q_3-Q_1}
    \label{eq:coef-ass-bowley}
\end{equation}
De forma análoga aos coeficientes anteriores, tem-se que:
\begin{itemize}
    \item se $A_{s_B}=0$, a distribuição é simétrica.
    \item se $A_{s_B}>0$, a distribuição é assimétrica positiva (à direita).
    \item se $A_{s_B}<0$, a distribuição é assimétrica negativa (à esquerda).
\end{itemize}
Usando a \Cref{eq:coef-ass-bowley} para determinar $A_{s_B}$ no
\Cref{eg:coleta-precos}, temos:
\begin{align*}
    \begin{cases}
        Q_1 = 179 \\
        Q_2 = 189 \\
        Q_3 = 199
    \end{cases} \implies
    &A_{s_B} = \frac{199+179-2 \cdot 189}{199-179} \implies \\
    &A_{s_B} = \frac{378-378}{20} \therefore \\
    &A_{s_B} = 0
\end{align*}
Nota-se que, para o terceiro coeficiente de assimetria determinado, chega-se à
terceira \footnote{Não sabemos interpretar o que está acontecendo.}
conclusão diferente.

\paragraph{Coeficiente de assimetria de Fisher}
É determinado a partir do terceiro momento em torno da média ($M_3$)
\parencite[p.~165]{favero}:
\begin{equation}
    g_1= \frac{n^2 \cdot M_3}{(n-1) \cdot (n-2) \cdot \sigma^3}
    \label{eq:coef-ass-fisher}
\end{equation}
em que
\begin{equation}
    M_3= \frac{1}{n} \cdot \sum_{i=1}^{n} (x_i - \overline{x})^3
    \label{eq:terceiro-momento-media}
\end{equation}
Façamos isso ``no braço'' para o \Cref{eg:coleta-precos}, iniciando com a
determinação de $M_3$ a partir da \Cref{eq:terceiro-momento-media}:
\begin{align*}
    M_3 &= \frac{1}{100} \cdot \Big( \\
    &(189.00-190.77)^3 + (195.00-190.77)^3 + (199.00-190.77)^3 + \\
    &+ \ldots + \\
    &(189.00-190.77)^3 + (199.00-190.77)^3 + (195.00-190.77)^3 \\
    &\Big) \implies \\
    M_3 &= \frac{-5.55 + 75.69 + 557.44 + \ldots + -5.55 + 557.44 + 75.69}{100} \implies \\
    M_3 &= \frac{33249.04}{100} \therefore \\
    M_3 &= 332.49
\end{align*}
Utilizando a \Cref{eq:coef-ass-fisher}, temos que $g_1$ é dado por:
\begin{align*}
    %
    g_1 &= \frac{23^2 \cdot 332.49}{22 \cdot 21 \cdot 15.621^3} \\
        &= \frac{529 \cdot 332.49}{462 \cdot 3811.768} \\
        &= \frac{175887.21}{1761036.97} \\
        &= 0.09988 \approx 0.1
\end{align*}
Em resumo, o que pudemos determinar até então:
\begin{itemize}
    \item Primeiro coeficiente de assimetria de Pearson:
        $A_{s_1}=-0.527<0$\\
        --- indica assimetria à esquerda.
    \item Segundo coeficiente de assimetria de Pearson:
        $A_{s_2}=0.340>0$\\
        --- indica assimetria à direita.
    \item Coeficiente de assimetria de Bowley:
        $A_{s_B}=0$\\
        -- indica distribuição simétrica.
    \item Coeficiente de assimetria de Fisher:
        $g_1=0.1>0$\\
        --- indica assimetria à direita.
\end{itemize}
Não há convergência total, mas os resultados de $A_{s_2}$ e $g_1$ mostram uma
certa consistência, indicando que há, de fato, uma distribuição assimétrica
positiva. Me parece \footnote{Confirmar se essa premissa é verdadeira.}
que, para o \Cref{eg:coleta-precos}, tanto o segundo coeficiente de Pearson
quanto o coeficiente de assimetria de Fisher são os mais adequados.

\subsubsection{Medidas de curtose}

A curtose pode ser definida como o grau de achatamento de uma distribuição de
frequências em relação a uma distribuição teórica que geralmente corresponde à
distribuição normal \parencite[p.~166]{favero}. Podemos ter:
\begin{itemize}
    \item \textbf{Mesocúrtica}: assemelha-se à distribuição normal --- não é
    muito achatada nem alongada.
    \item \textbf{Platicúrtica}: apresenta uma curva de frequências mais
    \emph{achatada} do que a curva normal.
    \item \textbf{Leptocúrtica}: apresenta uma curva de frequências mais
    \emph{alongada} do que a curva normal.
\end{itemize}

\paragraph{Coeficiente de curtose}
O coeficiente de curtose $k$, também chamado de \emph{coeficiente
percentílico de curtose} é um dos mais utilizados para medir o grau de
achatamento da curva de distribuição. Ele é dado por
\begin{equation}
    k = \frac{Q_3-Q_1}{2 \cdot (p_{90}-p_{10})}
    \label{eq:coef-curtose}
\end{equation}
em que $Q_1$ e $Q_3$ representam o primeiro e o terceiro quartis,
respectivamente, enquanto $p_{90}$ e $p_{10}$ representam o 90º e o 10º
percentis, respectivamente. Interpretamos o resultado da seguinte maneira:
\begin{itemize}
    \item $k<0.263$ indica curva \emph{leptocúrtica};
    \item $k=0.263$ indica curva \emph{mesocúrtica};
    \item $k>0.263$ indica curva \emph{platicúrtica}.
\end{itemize}

Vamos determinar $k$ para o \Cref{eg:coleta-precos}. Iniciamos com a
determinação dos percentis $p_{90}$ e $p_{10}$, usando a
\Cref{eq:pos-percentil}:
\begin{align*}
    P(p_{90}) &= \left[(100-1) \cdot \left(\frac{90}{100}\right)\right] + 1 \\
              &= 99 \cdot 0.9 + 1 \\
              &= 90.1 \therefore \\
    p_{90} &= 209
\end{align*}
\begin{align*}
    P(p_{10}) &= \left[(100-1) \cdot \left(\frac{10}{100}\right)\right] + 1 \\
              &= 99 \cdot 0.1 + 1 \\
              &= 10.9 \therefore \\
    p_{10} &= 169
\end{align*}
Dado que $Q_1=179$ e $Q_3=199$, tem-se que:
\begin{align*}
    k &= \frac{Q_3-Q_1}{2 \cdot (p_{90}-p_{10})} \\
      &= \frac{199-179}{2 \cdot (90.1 - 10.9)} \\
      &= \frac{20}{158.4} \therefore \\
    k &= 0.126
\end{align*}
De acordo com o critério anteriormente mencionado, a curva de distribuição é
\emph{leptocúrtica}.

\paragraph{Coeficiente de curtose de Fisher}
O coeficiente de curtose de Fisher $g_2$ é também bastante utilizado para medir
o grau de achatamento de uma curva de distribuição---no excel, a função
\texttt{CURT} utiliza tal coeficiente. Ele é calculado a partir do quarto
momento em torno da média ($M_4$)--- \emph{apud Maroco, 2014}
\parencite[p.~168]{favero}:
\revisar{Tenho que mexer nessa parte de \emph{apud}}
\begin{equation}
    g_2 = \frac{n^2 \cdot (n+1) \cdot M_4}{(n-1) \cdot (n-2) \cdot (n-3) \cdot \sigma^{4}} -
        3 \cdot \frac{(n-1)^2}{(n-2) \cdot (n-3)}
    \label{eq:coef-curtose-fisher}
\end{equation}
em que
\begin{equation}
    M_4 = \frac{1}{n} \cdot \sum_{i=1}^{n} (x_i - \overline{x})^{4}
    \label{eq:quarto-momento-media}
\end{equation}
Para a \Cref{eq:coef-curtose-fisher}, tem-se que:
\begin{itemize}
    \item $g_2<0$ indica curva \emph{platicúrtica}.
    \item $g_2=0$ indica curva \emph{mesocúrtica};
    \item $g_2>0$ indica curva \emph{leptocúrtica};
\end{itemize}
Determinemos $M_4$ para o \Cref{eg:coleta-precos}:
\begin{align*}
    M_4 &= \frac{1}{100} \cdot \Big( \\
    &(189.00-190.77)^4 + (195.00-190.77)^4 + (199.00-190.77)^4 + \\
    &+ \ldots + \\
    &(189.00-190.77)^4 + (199.00-190.77)^4 + (195.00-190.77)^4 \\
    &\Big) \implies \\
    M_4 &= \frac{9.82 + 320.16 + 4587.75 + \ldots + 9.82 + 4587.75 + 320.16}{100} \implies \\
    M_4 &= \frac{2087524.18}{100} \therefore \\
    M_4 &= 208754.24
\end{align*}
Com isso, pode-se utilizar a \Cref{eq:coef-curtose-fisher} para determinar
$g_2$:
\begin{align*}
    g_2 &= \frac{100^2 \cdot (100+1) \cdot 208754.24}{(100-1) \cdot (100-2) \cdot (100-3) \cdot 15.621^{4}} -
        3 \cdot \frac{(100-1)^2}{(100-2) \cdot (100-3)} \implies \\
    g_2 &= \frac{210,841,782,400.00}{56,036,155,804.04} -
        3 \cdot \frac{9,801}{9,506} \implies \\
    g_2 &= 3.76 - 3.09 \therefore \\
    g_2 &= 0.67
\end{align*}
De acordo com o critério acima mencionado, a curva de distribuição é
\emph{leptocúrtica}. Isso está de acordo com a determinação feita a partir do
coeficiente percentílico de curtose.

A \Cref{tab:medidas-resumo-estudo-caso} indica todas as medidas descritivas
abordadas na \Cref{sec:estatistica-descritiva}, para \Cref{eg:coleta-precos}.

\begin{table}[htpb]
    \centering
    \begin{tabular}{lcc}
    \toprule
                                      & Preço (R\$) \\
    \midrule
    Média                             & 190.77 \\
    Mediana                           & 189.00 \\
    Moda                              & 199.00 \\
    Primeiro quartil                  & 179.00 \\
    Terceiro quartil                  & 199.00 \\
    Amplitude                         & 90.00 \\
    Desvio médio                      & 12.08 \\
    Variância                         & 244.02 \\
    Desvio-padrão                     & 15.62 \\
    Erro padrão                       & 1.56 \\
    Coeficiente de variação           & 0.0819 \\
    Coeficiente de assimetria ($g_1$) & 0.1 \\
    Coeficiente de curtose ($g_2$)    & 0.67 \\
    \bottomrule
    \end{tabular}
    \caption{Estatísticas descritivas para o estudo de caso da
    \Cref{sec:estatistica-descritiva}}
    \label{tab:medidas-resumo-estudo-caso}
\end{table}
