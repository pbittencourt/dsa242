% <<< GENERAL PACKAGES -------------------------------------------------------

% Encoding and language
\usepackage[utf8]{inputenc}               % For UTF-8 encoding
\usepackage[T1]{fontenc}                  % For font encoding
\usepackage[brazil]{babel}                % Language-specific settings

% Page geometry and margins
\usepackage[a4paper,
            top=1in, bottom=1in, inner=1.5in, outer=1.5in
]{geometry}

% Quoting
\usepackage{csquotes}                     % For context-sensitive quotes

% Color and graphics
\usepackage[usenames,dvipsnames]{xcolor}  % Color management
\usepackage{graphicx}                     % For including graphics
\graphicspath{{graphics/}}                % Set the graphics path

% Float management and figures
\usepackage{float}                        % Improved float management
\usepackage{wrapfig}                      % For wrapping text around figures
\usepackage{subcaption}                   % For subfigures
\usepackage{booktabs}                     % For improved table formatting

% Mathematics packages
\usepackage{amsmath}                      % AMS math features
\usepackage{amsfonts}                     % AMS fonts
\usepackage{amssymb}                      % AMS symbols
\usepackage{amsthm}                       % Theorem environments
\usepackage{mathtools}                    % Additional math tools
\usepackage{empheq}                       % Enhanced equations
\usepackage{enumitem}                     % Custom lists
\usepackage{nicefrac}                     % Enhanced fractions

% Table cells merging
\usepackage{multicol}                     % For multi-column formatting
\usepackage{multirow}                     % For multi-row formatting

% Bibliography
\usepackage{xpatch}                       % Recommended by biblatex package
\usepackage[                              % Use biber for bibliography
    backend=biber,style=abnt
]{biblatex}
\addbibresource{references.bib}           % Bibliography file

% Miscellaneous
\usepackage{fancyhdr}                     % Format headers and footers
\usepackage{hyperref}                     % For hyperlinks
\usepackage{cleveref}                     % For clever references
\usepackage{bookmark}                     % For PDF bookmarks
\usepackage{siunitx}                      % For measure units and numbers
\usepackage{transparent}                  % Handle transparency in pdf figures
\usepackage{marginnote}                   % Allow margin notes
\usepackage{tikz}
\usepackage{tikz-cd}
\usepackage{pgfplots}
\usepackage{xifthen}
\usepackage{xargs}
\usepackage[colorinlistoftodos,prependcaption,textsize=small]{todonotes}
\usepackage{thmtools}
\usepackage[framemethod=TikZ]{mdframed}

% >>> ------------------------------------------------------------------------

% <<< HEADERS AND FOOTERS ----------------------------------------------------

\pagestyle{fancy}
\fancyhf{}
\fancyhead[LO]{\nouppercase{\leftmark}}
\fancyhead[RE]{\nouppercase{\rightmark}}
\fancyfoot[C]{\thepage}

% >>> ------------------------------------------------------------------------

% <<< REFERENCES AND NUMBERING -----------------------------------------------

% package `cleveref` don't have support for pt-BR
\crefname{equation}{equação}{equações}
\Crefname{equation}{Equação}{Equações}
\crefname{figure}{figura}{figuras}
\Crefname{figure}{Figura}{Figuras}
\crefname{section}{seção}{seções}
\Crefname{section}{Seção}{Seções}
\crefname{subsection}{sub-seção}{sub-seções}
\Crefname{subsection}{Sub-seção}{Sub-seções}
\crefname{subsubsection}{sub-sub-seção}{sub-sub-seções}
\Crefname{subsubsection}{Sub-sub-seção}{Sub-sub-seções}
\crefname{table}{tabela}{tabelas}
\Crefname{table}{Tabela}{Tabelas}
\crefname{eg}{exemplo}{exemplos}
\Crefname{eg}{Exemplo}{Exemplos}
\crefname{definition}{definição}{definições}
\Crefname{definition}{Definição}{Definições}

% set how we will number equations, figures, tables and so on
\numberwithin{equation}{section}
\numberwithin{figure}{section}
\numberwithin{table}{section}

% start every section in a new page
\let\oldsection\section % sometimes we want to use the default style ...
\renewcommand\section{\clearpage\oldsection}

% number til sub sub section
\setcounter{secnumdepth}{3}
% include sub sub sections in TOC
\setcounter{tocdepth}{3}

% >>> ------------------------------------------------------------------------

% <<< SI UNITS ---------------------------------------------------------------

\sisetup{locale = US}
\sisetup{                    % FIXME isso ainda não funciona
  detect-all,                % Apply formatting even to numbers outside math environments
  output-decimal-marker={,}, % Use comma as decimal separator
  group-separator={.},       % Use dot as thousands separator
  group-minimum-digits=4     % Only apply thousands separator for numbers with 4 or more digits
}
%\DeclareSIUnit[number-unit-product = {}]{\decimalnumber}{}

% >>> ------------------------------------------------------------------------

% <<< TIKZ -------------------------------------------------------------------

\usetikzlibrary{
    intersections, angles, quotes, calc, positioning, arrows.meta
}
\pgfplotsset{compat=1.13}
\tikzset{
    force/.style={thick, {Circle[length=2pt]}-stealth, shorten <=-1pt}
}

% >>> ------------------------------------------------------------------------

% <<< ANNOTATIONS ------------------------------------------------------------

% Insert lesson info on margins
\setlength{\marginparwidth}{2.5cm}
\makeatletter % Permite o uso do "@"
\def\testdateparts#1{\dateparts#1\relax}
\def\dateparts#1 #2 #3 #4 #5\relax{
    \marginpar{\small\textsf{\mbox{#1 #2 #3 #4}}}
}
\def\@lesson{}%
\newcommand{\lesson}[3]{
    \def\@lesson{Aula #1}%
    \marginpar{\small\textsf{\mbox{\@lesson}}}
    \testdateparts{#2}
}
\makeatother % Restaura o uso padrão do "@"

% Insert TODO notes on margins
\newcommandx{\conferir}[2][1=]{\todo[linecolor=RawSienna,backgroundcolor=RawSienna!25,bordercolor=RawSienna,#1]{#2}}
\newcommandx{\revisar}[2][1=]{\todo[linecolor=NavyBlue,backgroundcolor=NavyBlue!25,bordercolor=NavyBlue,#1]{#2}}
\newcommandx{\info}[2][1=]{\todo[linecolor=ForestGreen,backgroundcolor=ForestGreen!25,bordercolor=ForestGreen,#1]{#2}}

% >>> ------------------------------------------------------------------------

% <<< THEOREMS STYLES --------------------------------------------------------

% define margins for theorems environments, via mdfsetup
\mdfsetup{skipabove=1em,skipbelow=1em}

% define colored symbols
% FIXME
% \newcommand{\blueqed}{\textcolor{NavyBlue}{\blacksquare}}
% \newcommand{\greenqed}{\textcolor{ForestGreen}{\blacksquare}}

% ---------------------------------------
% somente título, mesma linha do conteúdo
%
\declaretheoremstyle[ % thmredhead
    headfont=\bfseries\sffamily\color{RawSienna},
    headindent=0pt,
    headpunct={},
    spaceabove=2em,
]{thmredhead}
\declaretheoremstyle[ % thmgreenhead
    headfont=\bfseries\sffamily\color{ForestGreen},
    headindent=0pt,
    headpunct={},
    spaceabove=2em,
]{thmgreenhead}
\declaretheoremstyle[ % thmbluehead
    headfont=\bfseries\sffamily\color{NavyBlue},
    headindent=0pt,
    headpunct={},
    spaceabove=2em,
]{thmbluehead}

% --------------------------------------------------
% título em linha separada do conteúdo, QED ao final
%
\declaretheoremstyle[ % thmredblockqed
    headfont=\bfseries\itshape\sffamily\color{RawSienna},
    headindent=0pt,
    headpunct={},
    postheadspace=\newline,
    qed=\qedsymbol
]{thmredblockqed}
\declaretheoremstyle[ % thmgreenblockqed
    headfont=\bfseries\itshape\sffamily\color{ForestGreen},
    headindent=0pt,
    headpunct={},
    postheadspace=\newline,
    qed=\qedsymbol
]{thmgreenblockqed}
\declaretheoremstyle[ % thmblueblockqed
    headfont=\bfseries\itshape\sffamily\color{NavyBlue},
    headindent=0pt,
    headpunct={},
    postheadspace=\newline,
    qed=\qedsymbol
]{thmblueblockqed}

% -------------------------------
% borda à esquerda e cor de fundo
%
\declaretheoremstyle[ % thmredbox
    headfont=\bfseries\sffamily\color{RawSienna!70!black},
    bodyfont=\normalfont,
    % headpunct={}, postheadhook=\hspace{0.5em}, spacebelow=1em,
    mdframed={
        backgroundcolor=RawSienna!5,
        linewidth=2pt,
        linecolor=RawSienna,
        rightline=false, topline=false, bottomline=false
    }
]{thmredbox}
\declaretheoremstyle[ % thmgreenbox
    headfont=\bfseries\sffamily\color{ForestGreen!70!black},
    bodyfont=\normalfont,
    mdframed={
        backgroundcolor=ForestGreen!5,
        linewidth=2pt,
        linecolor=ForestGreen,
        rightline=false, topline=false, bottomline=false,
    }
]{thmgreenbox}
\declaretheoremstyle[ % thmbluebox
    headfont=\bfseries\sffamily\color{NavyBlue!70!black},
    bodyfont=\normalfont,
    mdframed={
        backgroundcolor=NavyBlue!5,
        linewidth=2pt,
        linecolor=NavyBlue,
        rightline=false, topline=false, bottomline=false
    }
]{thmbluebox}

% ---------------------------------
% borda à esquerda, sem cor de fundo
%
\declaretheoremstyle[ % thmredline
    headfont=\bfseries\sffamily\color{RawSienna!70!black},
    bodyfont=\normalfont,
    mdframed={
        linewidth=2pt,
        linecolor=RawSienna,
        rightline=false, topline=false, bottomline=false
    }
]{thmredline}
\declaretheoremstyle[ % thmgreenline
    headfont=\bfseries\sffamily\color{ForestGreen!70!black},
    bodyfont=\normalfont,
    mdframed={
        linewidth=2pt,
        linecolor=ForestGreen,
        rightline=false, topline=false, bottomline=false
    }
]{thmgreenline}
\declaretheoremstyle[ % thmblueline
    headfont=\bfseries\sffamily\color{NavyBlue!70!black},
    bodyfont=\normalfont,
    mdframed={
        linewidth=2pt,
        linecolor=NavyBlue,
        rightline=false, topline=false, bottomline=false
    }
]{thmblueline}

% >>> ------------------------------------------------------------------------

% <<< THEOREMS DEFINITIONS ---------------------------------------------------

% numbered environments
\declaretheorem[name=Teorema, numberwithin=chapter, style=thmgreenbox]{theorem}
\declaretheorem[name=Definição, numberwithin=chapter, style=thmbluebox]{definition}
\declaretheorem[name=Exemplo, numberwithin=chapter, style=thmbluehead]{eg}
\declaretheorem[name=Estudo de caso, numberwithin=chapter, style=thmredbox]{case}
\declaretheorem[name=Exercicio, numberwithin=chapter, style=thmgreenhead]{exercise}

% non-numbered environments
\declaretheorem[numbered=no, name=Corolário]{corollary}
\declaretheorem[numbered=no, name=Observação]{remark}
\declaretheorem[numbered=no, name=Nota]{note}
\declaretheorem[numbered=no, name=Demonstração, style=thmredblockqed]{demo}
\declaretheorem[numbered=no, name=Solução, style=thmblueblockqed]{sol}
\declaretheorem[numbered=no, name=Resolução, style=thmgreenblockqed]{resolucao}

% apply styles to already defined environments
\declaretheorem[numbered=no, name=Prova, style=thmredblockqed]{replacementproof}
\renewenvironment{proof}[1][\proofname]{\begin{replacementproof}}{\end{replacementproof}}

% >>> ------------------------------------------------------------------------

%vim:fileencoding=utf-8:foldmethod=marker:foldmarker:<<<,>>>
