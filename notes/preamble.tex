% <<< GENERAL PACKAGES -------------------------------------------------------

% Encoding and language
\usepackage[utf8]{inputenc}               % For UTF-8 encoding
\usepackage[T1]{fontenc}                  % For font encoding
\usepackage[brazil]{babel}                % Language-specific settings

% Quoting
\usepackage{csquotes}                     % For context-sensitive quotes

% Color and graphics
\usepackage[usenames,dvipsnames]{xcolor}  % Color management
\usepackage{graphicx}                     % For including graphics
\graphicspath{{graphics/}}                % Set the graphics path

% Float management and figures
\usepackage{float}                        % Improved float management
\usepackage{wrapfig}                      % For wrapping text around figures
\usepackage{subcaption}                   % For subfigures
\usepackage{booktabs}                     % For improved table formatting

% Mathematics packages
\usepackage{amsmath}                      % AMS math features
\usepackage{amsfonts}                     % AMS fonts
\usepackage{amssymb}                      % AMS symbols
\usepackage{amsthm}                       % Theorem environments
\usepackage{mathtools}                    % Additional math tools
\usepackage{empheq}                       % Enhanced equations
\usepackage{enumitem}                     % Custom lists
\usepackage{nicefrac}                     % Enhanced fractions

% Table cells merging
\usepackage{multicol}                     % For multi-column formatting
\usepackage{multirow}                     % For multi-row formatting

% Bibliography
\usepackage{xpatch}                       % Recommended by biblatex package
\usepackage[                              % Use biber for bibliography
    backend=biber,style=abnt
]{biblatex}
\addbibresource{references.bib}           % Bibliography file

% Miscellaneous
\usepackage{fancyhdr}           % Format headers and footers
\usepackage{hyperref}                     % For hyperlinks
\usepackage{cleveref}                     % For clever references
\usepackage{bookmark}                     % For PDF bookmarks
\usepackage{siunitx}                      % For measure units and numbers
\usepackage{transparent}                  % Handle transparency in pdf figures
\usepackage{tikz}
\usepackage{tikz-cd}
\usepackage{pgfplots}
\usepackage{xifthen}
\usepackage{xargs}
\usepackage[colorinlistoftodos,prependcaption,textsize=small]{todonotes}
\usepackage{thmtools}
\usepackage[framemethod=TikZ]{mdframed}

% >>> ------------------------------------------------------------------------

% <<< HEADERS AND FOOTERS ----------------------------------------------------

\pagestyle{fancy}
\fancyhf{}
\fancyhead[LO]{\nouppercase{\leftmark}}
\fancyhead[RE]{\nouppercase{\rightmark}}
\fancyfoot[C]{\thepage}

% >>> ------------------------------------------------------------------------

% <<< REFERENCES AND NUMBERING -----------------------------------------------

% package `cleveref` don't have support for pt-BR
\crefname{equation}{equação}{equações}
\Crefname{equation}{Equação}{Equações}
\crefname{figure}{figura}{figuras}
\Crefname{figure}{Figura}{Figuras}
\crefname{section}{seção}{seções}
\Crefname{section}{Seção}{Seções}
\crefname{subsection}{sub-seção}{sub-seções}
\Crefname{subsection}{Sub-seção}{Sub-seções}
\crefname{subsubsection}{sub-sub-seção}{sub-sub-seções}
\Crefname{subsubsection}{Sub-sub-seção}{Sub-sub-seções}
\crefname{table}{tabela}{tabelas}
\Crefname{table}{Tabela}{Tabelas}
\crefname{eg}{exemplo}{exemplos}
\Crefname{eg}{Exemplo}{Exemplos}
\crefname{definition}{definição}{definições}
\Crefname{definition}{Definição}{Definições}

% set how we will number equations, figures, tables and so on
\numberwithin{equation}{section}
\numberwithin{figure}{section}
\numberwithin{table}{section}

% start every section in a new page
\let\oldsection\section % sometimes we want to use the default style ...
\renewcommand\section{\clearpage\oldsection}

% number til sub sub section
\setcounter{secnumdepth}{3}
% include sub sub sections in TOC
\setcounter{tocdepth}{3}

% >>> ------------------------------------------------------------------------

% <<< SI UNITS ---------------------------------------------------------------

\sisetup{locale = US}
\sisetup{                    % FIXME isso ainda não funciona
  detect-all,                % Apply formatting even to numbers outside math environments
  output-decimal-marker={,}, % Use comma as decimal separator
  group-separator={.},       % Use dot as thousands separator
  group-minimum-digits=4     % Only apply thousands separator for numbers with 4 or more digits
}
%\DeclareSIUnit[number-unit-product = {}]{\decimalnumber}{}

% >>> ------------------------------------------------------------------------

% <<< TIKZ -------------------------------------------------------------------

\usetikzlibrary{
    intersections, angles, quotes, calc, positioning, arrows.meta
}
\pgfplotsset{compat=1.13}
\tikzset{
    force/.style={thick, {Circle[length=2pt]}-stealth, shorten <=-1pt}
}

% >>> ------------------------------------------------------------------------

% <<< ANNOTATIONS ------------------------------------------------------------

% Insert lesson info on margins
\setlength{\marginparwidth}{2.5cm}
\def\testdateparts#1{\dateparts#1\relax}
\def\dateparts#1 #2 #3 #4 #5\relax{
    \marginpar{\small\textsf{\mbox{#1 #2 #3 #4}}}
}
\def\@lesson{}%
\newcommand{\lesson}[3]{
    %\ifthenelse{\isempty{#3}}{%
    %    \def\@lesson{Aula #1}%
    %}{%
    %    \def\@lesson{Aula #1: #3}%
    %}%
    %\subsection*{\@lesson}
    \def\@lesson{Aula #1}%
    \marginpar{\small\textsf{\mbox{\@lesson}}}
    \testdateparts{#2}
}

% Insert TODO notes on margins
\newcommandx{\conferir}[2][1=]{\todo[linecolor=RawSienna,backgroundcolor=RawSienna!25,bordercolor=RawSienna,#1]{#2}}
\newcommandx{\revisar}[2][1=]{\todo[linecolor=NavyBlue,backgroundcolor=NavyBlue!25,bordercolor=NavyBlue,#1]{#2}}
\newcommandx{\info}[2][1=]{\todo[linecolor=ForestGreen,backgroundcolor=ForestGreen!25,bordercolor=ForestGreen,#1]{#2}}

% >>> ------------------------------------------------------------------------

% <<< THEOREMS STYLES --------------------------------------------------------

%\makeatletter % não sei pra isso serve aqui ...

% define margins for theorems environments, via mdfsetup
\mdfsetup{skipabove=1em,skipbelow=1em}

\declaretheoremstyle[
    headfont=\bfseries\sffamily\color{NavyBlue},
    % headformat=\NAME~\NUMBER,
    headindent=0pt,
    headpunct={},
    spaceabove=2em,
]{thmbluehead}
\declaretheoremstyle[
    headfont=\bfseries\itshape\sffamily\color{NavyBlue},
    % headformat=\NAME,
    headindent=0pt,
    headpunct={},
    postheadspace=\newline,
    qed=\textcolor{NavyBlue}{\blacksquare}
]{thmblueblockqed}
\declaretheoremstyle[
    headfont=\bfseries\sffamily\color{ForestGreen},
    % headformat=\NAME~\NUMBER,
    headindent=0pt,
    headpunct={},
    spaceabove=2em,
]{thmgreenhead}
\declaretheoremstyle[
    headfont=\bfseries\itshape\sffamily\color{ForestGreen},
    % headformat=\NAME,
    headindent=0pt,
    headpunct={},
    postheadspace=\newline,
    qed=\textcolor{ForestGreen}{\blacksquare}
]{thmgreenblockqed}
\declaretheoremstyle[
    headfont=\bfseries\sffamily\color{ForestGreen!70!black}, bodyfont=\normalfont,
    mdframed={
        linewidth=2pt,
        rightline=false, topline=false, bottomline=false,
        linecolor=ForestGreen, backgroundcolor=ForestGreen!5,
    }
]{thmgreenbox}
\declaretheoremstyle[
    headfont=\bfseries\sffamily\color{ForestGreen!70!black}, bodyfont=\normalfont,
    mdframed={
        linewidth=2pt,
        rightline=false, topline=false, bottomline=false,
        linecolor=ForestGreen
    }
]{thmgreenline}
\declaretheoremstyle[
    headfont=\bfseries\sffamily\color{NavyBlue!70!black}, bodyfont=\normalfont,
    mdframed={
        linewidth=2pt,
        rightline=false, topline=false, bottomline=false,
        linecolor=NavyBlue, backgroundcolor=NavyBlue!5,
    }
]{thmbluebox}
\declaretheoremstyle[
    headfont=\bfseries\sffamily\color{NavyBlue!70!black}, bodyfont=\normalfont,
    mdframed={
        linewidth=2pt,
        rightline=false, topline=false, bottomline=false,
        linecolor=NavyBlue
    }
]{thmblueline}
\declaretheoremstyle[
    headfont=\bfseries\sffamily\color{RawSienna!70!black},
    bodyfont=\normalfont,
    headpunct={},
    postheadhook=\hspace{0.5em},
    spacebelow=1em,
    mdframed={
        linewidth=2pt,
        rightline=false, topline=false, bottomline=false,
        linecolor=RawSienna, backgroundcolor=RawSienna!5,
    }
]{thmredbox}
\declaretheoremstyle[
    headfont=\bfseries\sffamily\color{RawSienna!70!black}, bodyfont=\normalfont,
    numbered=no,
    mdframed={
        linewidth=2pt,
        rightline=false, topline=false, bottomline=false,
        linecolor=RawSienna, backgroundcolor=RawSienna!1,
    },
    qed=\qedsymbol
]{thmproofbox}
\declaretheoremstyle[
    headfont=\bfseries\sffamily\color{NavyBlue!70!black}, bodyfont=\normalfont,
    numbered=no,
    mdframed={
        linewidth=2pt,
        rightline=false, topline=false, bottomline=false,
        linecolor=NavyBlue, backgroundcolor=NavyBlue!1,
    },
]{thmexplanationbox}

% >>> ------------------------------------------------------------------------

% <<< THEOREMS DEFINITIONS ---------------------------------------------------

% numbered environments
\declaretheorem[name=Teorema, numberwithin=chapter, style=thmgreenbox]{theorem}
\declaretheorem[name=Definição, numberwithin=chapter, style=thmbluebox]{definition}
\declaretheorem[name=Proposição, numberwithin=chapter, style=thmexplanationbox]{prop}
\declaretheorem[name=Exemplo, numberwithin=chapter, style=thmbluehead]{eg}
\declaretheorem[name=Estudo de caso, numberwithin=chapter, style=thmredbox]{case}
\declaretheorem[name=Exercicio, numberwithin=chapter, style=thmgreenhead]{exercise}

% These environments do not have top margins—they are designed to be "glued" to
% the preceding environment. For example, when we state a definition, we
% typically follow it with a proof of that definition.
\declaretheorem[name=Prova, style=thmproofbox]{replacementproof}
\renewenvironment{proof}[1][\proofname]{\vspace{-2em}\begin{replacementproof}}{\end{replacementproof}}
\declaretheorem[name=Demonstração, style=thmproofbox]{replacementdemo}
\newenvironment{demo}[1][\proofname]{\vspace{-2em}\begin{replacementdemo}}{\end{replacementdemo}}

% non-numbered environments
\declaretheorem[numbered=no, name=Corolário]{corollary}
\declaretheorem[numbered=no, name=Observação]{remark}
\declaretheorem[numbered=no, name=Nota]{note}
\declaretheorem[numbered=no, name=Solução, style=thmblueblockqed]{egsol}
\declaretheorem[numbered=no, name=Solução, style=thmblueblockqed]{sol}
\declaretheorem[numbered=no, name=Resolução, style=thmgreenblockqed]{resolucao}

% >>> ------------------------------------------------------------------------

%vim:fileencoding=utf-8:foldmethod=marker:foldmarker:<<<,>>>
